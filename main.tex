\documentclass{article}
\bibliographystyle{main}
\usepackage[UTF8]{ctex}

\usepackage{graphicx} 
\usepackage{subfigure}
\usepackage{bibentry}
\usepackage{booktabs}
\usepackage[mathscr]{eucal}
\usepackage{amsfonts}
\usepackage{amsmath}
\usepackage{amssymb}
\usepackage{amsthm}
\usepackage{bm}
\usepackage{natbib}
\usepackage{bbm}
\usepackage{cases}
\usepackage{footmisc}
\usepackage{geometry}
\usepackage[colorlinks]{hyperref}

\geometry{a4paper, scale=0.7}

\newtheorem{definition}{定义}[section]
\newtheorem{theorem}{定理}[section]
\newtheorem{lemma}{引理}[section]
\newtheorem{remark}{注}[section]

\usepackage[ruled,linesnumbered]{algorithm2e}
\renewcommand{\algorithmcfname}{算法}
\author{WangFangyikang}


\allowdisplaybreaks

\begin{document}
    \title{SDE笔记}
    \maketitle
    \tableofcontents


    \newpage
    %\section{Computational Complexity}
我们定义$L,(P,r,\gamma)$是确定一个MDP所需的bit-size,假定算术运算+,-,$\times$,$\div$都使用单位时间(unit time)。我们希望找到一个算法,它能找到在$L,(P,r,\gamma)$、$|mathcal{S}|$、$|mathcal{A}|$的多项式时间内找到最优策略。
如果一个算法能在$|S|$、$|A|$的多项式时间(与$L,(P,r,\gamma)$无关)内找到最优策略,则称该算法为strongly polynomial。
\subsection{Value Iteration}
\begin{lemma}[收缩性]
    对任意两个向量$Q$,$Q'\in \mathbb{R}^{|mathcal{S}||mathcal{A}|}$,$\Vert \tau Q-\tau Q' \Vert_\infty\leq\gamma\Vert Q-Q' \Vert_\infty$。
    这说明了$\tau$是一个收缩映射。
 \end{lemma}
\begin{proof}
    首先证明对于所有的$s\in \mathcal{S},|V_Q(s)-V_{Q'(s)}|\leq \underset{a\in \mathcal{A}}{max}|Q(s,a)-Q'(s,a)|$。\\
    不妨设$V_Q(s)>V_{Q'}(s)$(另一半是对称的),记$a^*=\underset{a}{argmax}Q(s,a)$。\\
    $|V_Q(s)-V_{Q'(s)}|=\underset{a\in\mathcal{A}}{max}Q(s,a)-\underset{a\in\mathcal{A}}{max}Q'(s,a)\leq Q(s,a^*)-Q'(s,a^*)\leq\underset{a\in\mathcal{A}}{max}|Q(s,a)-Q'(s,a)|$
    \begin{equation}
        \begin{aligned}
            \Vert \tau Q-\tau Q' \Vert_\infty &= \Vert r+\gamma PV_Q-r-\gamma PV_{Q'} \Vert_\infty\\
            &= \gamma \Vert P(V_Q-V_{Q'}) \Vert_\infty\\
            &\leq \gamma \Vert V_Q-V_{Q'} \Vert_\infty\\
            &= \gamma \underset{s}{\max}|V_Q-V_{Q'}|\\
            &\leq \underset{s}{\max}\underset{a}{\max}|Q(s,a)-Q'(s,a)|\\
            &= \gamma\Vert Q-Q' \Vert_\infty
        \end{aligned}
    \end{equation}
\end{proof}
由该引理可见,经历了k次迭代后,$\Vert Q_k-Q^* \Vert_\infty\leq\gamma^k\Vert Q_0-Q^*\Vert_\infty$,因此$\underset{k\to \infty}{\lim}Q_k-Q^*=0$,算法收敛。

\begin{lemma}[Q-Error Amplification)]
    对任意向量$Q\in \mathbb{R}^{|mathcal{S}||mathcal{A}|},V^{\pi_Q}\geq V^*-\frac{2\Vert Q-Q^*\Vert_\infty}{1-\gamma}\mathbbm{1}$
    \\其中$\mathbbm{1}$为全1向量。
 \end{lemma}
\begin{proof}
    对于固定的$s$以及$a=\pi_Q(s),(\pi_Q(s)=\underset{a\in \mathcal{A}}{argmax}Q(s,a))$
    \begin{equation}
        \begin{aligned}
            V^*(s)-V^{\pi_Q}(s)&=Q^*(s,\pi^*(s))-Q^{\pi_Q}(s,a)\\
            &=Q^*(s,\pi^*(s))-Q^*(s,a)+Q^*(s,a)-Q^(\pi_Q)(s,a)\\
            &=Q^*(s,\pi^*(s))-Q^*(s,a)+\gamma \mathbb{E}_{s'\sim P(\cdot |s,a)}[V^*(s')-V^{\pi_Q}(s')]\\
            &\leq Q^*(s,\pi^*(s))-Q(s,\pi^*(s))+Q(s,a)-Q^*(s,a)+\gamma \mathbb{E}_{s'\sim P(\cdot |s,a)}[V^*(s')-V^{\pi_Q}(s')]\\
            &\leq 2\Vert Q-Q^* \Vert_\infty+\gamma\Vert V^*-V^{\pi_Q}\Vert_\infty
        \end{aligned}
    \end{equation}
\end{proof}

\begin{theorem}[Q-value iteration convergence]
    设$Q^{(0)}=0,Q^{(k+1)}=\tau Q^{(k)},k=0,1,\ldots \pi^{(k)=pi_{Q^{(k)}}}$,当$k\geq \frac{\log \frac{2}{(1-\gamma )^2\epsilon}}{1-\gamma}$,\\
    $V^{\pi^{(k)}}\geq V^*-\epsilon \mathbbm{1}$,即k次迭代后$V^{\pi^{(k)}}$和$V^*$非常接近。
\end{theorem}
\begin{proof}
    $\Vert Q^{(k)}-Q^*\Vert_\infty=\Vert \tau ^kQ^{(0)}-\tau ^kQ^*\Vert_\infty \leq\gamma^k\Vert Q^{(0)}-Q^*\Vert_\infty=(1-()1-\gamma)^k\leq \frac{exp(-(1-\gamma)k)}{1-\gamma} $
\end{proof}

\subsection{Value Iteration}
策略迭代算法从任意一个策略$\pi_0$出发,并对k=0,1,2,$\ldots $重复接下来的两步:
1.策略评估。计算$Q^{\pi_k}$\\
2.策略提升。更新策略:$\pi_{k+1}=\pi_{Q^{\pi_k}}$,即$\pi_{k+1}(s)=\underset{a\in\mathcal{A}}{argmax}Q^{\pi_k}(s,a)$

\begin{lemma}
    1.$Q^{\pi_{k+1}}\geq \tau Q^{\pi_k}\geq Q^{\pi_k}$\\
    2.$\Vert Q^{\pi_{k+1}}-Q^* \Vert_\infty\leq\gamma\Vert Q^{\pi_k}-Q^* \Vert_\infty$
\end{lemma}
\begin{proof}
    首先证明$\tau Q^{\pi_k}\geq Q^{\pi_k}$。注意到策略迭代中的策略都是确定策略,䫅对于所有$k,s$,$V^{\pi_k}(s)=Q^{\pi_k}(s,\pi_k(s))$。\\
    \begin{equation}
        \begin{aligned}
            \tau Q^{\pi_k}(s,a)&=r(s,a)+\gamma \mathbb{E}_{s'\sim P(\cdot |s,a)}[\underset{a'}{\max}Q^{\pi_k}(s',a')]\\
            &\geq r(s,a)+\gamma\mathbb{E}_{s'\sim P(\cdot |s,a)}[Q^{\pi_k}(s',\pi_k(s'))]=Q^{\pi_k}(s,a)
        \end{aligned}
    \end{equation}
    \\再证明$Q^{\pi_{k+1}}\geq \tau Q^{\pi_k}$,这需要先证明$Q^{\pi_{k+1}}\geq Q^{\pi_k}$:\\
    $Q^{\pi_k}=r+\gamma P^{\pi_k}Q^{\pi_k}\leq r+\gamma P^{\pi_{k+1}}Q^{\pi_k}\leq \sum_{t=0}^{\infty}\gamma^t(P^{\pi_{k+1}})^tr=Q^{\pi_{k+1}}$\\
    第一个不等式是因为$\pi_{k+1}$是greedy policy,所以一定比$\pi_{k}$更好,第二个不等式由递归得到。\\因此
    \begin{equation}
        \begin{aligned}
            Q^{\pi_{k+1}}(s,a)&=r(s,a)+\gamma \mathbb{E}_{s'\sim P(\cdot |s,a)}[Q^{\pi_{k+1}}(s',\pi_{k+1}(s'))]\\
            &\geq r(s,a)+\gamma \mathbb{E}_{s'\sim P(\cdot |s,a)}[Q^{\pi_{k}}(s',\pi_{k+1}(s'))]\\
            &=r(s,a)+\gamma \mathbb{E}_{s'\sim P(\cdot |s,a)}[\underset{a'}{\max}Q^{\pi_k}(s',a')]=\tau Q^{\pi_k}(s,a)
        \end{aligned}
    \end{equation}
    (1)式证明完成。
    下面证明(2)式:\\
    $\Vert Q^*-Q^{\pi_{k+1}} \Vert_\infty\leq\Vert Q^*-\tau Q^{\pi_{k}} \Vert_\infty =\Vert \tau Q^*-\tau Q^{\pi_{k}} \Vert_\infty\leq\gamma\Vert Q^*-Q^{\pi_k} \Vert_\infty$\\
    证明完成。
\end{proof}

\begin{theorem}[Policy iteration convergence]
    设$Q^{\pi_0}=0,\pi_0$为任意初始策略。当$k\geq \frac{\log\frac{1}{(1-\gamma )\epsilon}}{1-\gamma} $,第k个策略有这样的performance bound:\\
    $Q^{\pi_k}\geq Q^*-\epsilon $
\end{theorem}

\begin{proof}
    \begin{equation}
        \begin{aligned}
            \Vert Q^*-Q^{\pi_k}\Vert_\infty&\leq\gamma\Vert Q^*-Q^{\pi_{k-1}}\Vert_\infty
            &\leq \gamma^k\Vert Q^*-Q^{\pi_0}\Vert_\infty
            &=\gamma^k\Vert Q^\Vert_\infty
            &=(1-(1-\gamma))^k\Vert Q^\Vert_\infty
            &\leq \frac{\exp (-(1-\gamma)k)}{1-\gamma} 
        \end{aligned}
    \end{equation}
\end{proof}


\subsection{Linear Programming Approach}
线性规划的方法可以在严格多项式时间内解决问题。\\
\subsubsection{原始问题}
最初的想法是求解\\
$\qquad\underset{V\in \mathbb{R}^{|\mathcal{S}|}}{\min}\qquad{\sum_s\mu (s)V(s)}$\\
subject to  $V(s)\geq\underset{a\in \mathcal{A}}{\max}[r(s,a)+\gamma\underset{s'}{\sum}{P(s'|s,a)V(s')}] \qquad \forall s\in\mathcal{S}$\\
但这不是LP问题。将其转化为LP问题,得到:\\
$\qquad\underset{V\in \mathbb{R}^{|\mathcal{S}|}}{\min}\qquad{\sum_s\mu (s)V(s)}$\\
subject to  $V(s)\geq r(s,a)+\gamma\underset{s'}{\sum}{P(s'|s,a)V(s')} \qquad \forall a\in\mathcal{A},s\in\mathcal{S} $\\
其中,$\mu (s)$是初始状态分布。如果$\mu$ has full support,那么该问题的唯一解就是$V^*(s)$。
\begin{proof}
    证明利用了$\tau$的单调性,即$V \geq V^{'}$时,有$\tau V \geq \tau V^{'}$。\\
    令$V'=\tau V$,则$\tau V\geq\tau V'=\tau^2V$。迭代得到:$V\geq\tau^\infty=V^*$。\\
    因此该约束条件下得到的解都是$V\geq V^*$的情况,由于目标函数是求$\qquad\underset{V\in \mathbb{R}^{|\mathcal{S}|}}{\min}\qquad{\sum_s\mu (s)V(s)}$,最终得到的解即$V=V^*$。
\end{proof}
\subsubsection{对偶问题}
对每一个LP都存在一个对偶问题,原问题的决策变量对应对偶问题的约束条件,原问题的约束条件对应对偶问题的决策变量。\\
对于固定的策略$\pi$,定义关于状态和动作的visitation measure:
\begin{equation}
    d^{\pi}_{s_0}(s,a)=(1-\gamma)\sum_{t=0}^{\infty}\gamma^tPr^\pi (s_t=s,a_t=a|s_0)
\end{equation}
其中$Pr^\pi (s_t=s,a_t=a|s_0)$是从状态$s_0$出发,经过策略$\pi$,到达$s_t=s,a_t=a$的概率。并记$d^{\pi}_{\mu}(s,a)=\mathbb{E}_{s_0\sim\mu}[d^{\pi}_{s_0}(s,a)]$\\
对于任意的状态$s$有:
\begin{equation}
    \underset{a}{\sum}d^{\pi}_{\mu}(s,a)=(1-\gamma)\mu(s)+\gamma\underset{s',a'}{\sum}P(s|s',a')d^{\pi}_{\mu}(s',a')
\end{equation}

\begin{proof}
    左边:\\
    \begin{equation}
        \begin{aligned}
            \underset{a}{\sum}d^{\pi}_{\mu}(s,a)&=d^{\pi}_{\mu}(s)\\
           &=\underset{s_0\sim \mu}{\mathbb{E}}[d^{\pi}_{s_0}(s)]\\
           &=\underset{s_0\sim \mu}{\mathbb{E}}[(1-\gamma)\sum_{t=0}^{\infty}\gamma^tPr^{\pi}(s_t=s|s_0)]
        \end{aligned}
    \end{equation}
    右边:
    \begin{equation}
        \begin{aligned}
            &\gamma\underset{s',a'}{\sum}P(s|s',a')d^{\pi}_{\mu}(s',a')+(1-\gamma)\mu(s)\\
            =&\gamma\underset{s',a'}{\sum}P(s|s',a')\underset{s_0\sim \mu}{\mathbb{E}}[(1-\gamma)\sum_{t=0}^{\infty}\gamma^tPr^{\pi}(s_t=s',a_t=a'|s_0)]+(1-\gamma)\mu(s)\\
            =&\underset{s_0\sim \mu}{\mathbb{E}}[(1-\gamma)\sum_{t=0}^{\infty}\sum_{s',a'}\gamma^{t+1}P(s|s',a')Pr^{\pi}(s_t=s',a_t=a'|s_0)]+(1-\gamma)\mu(s)\\
            =&\underset{s_0\sim \mu}{\mathbb{E}}[(1-\gamma)\sum_{t=0}^{\infty}\sum_{s',a'}\gamma^{t+1}Pr^{\pi}(s|s_0)]+(1-\gamma)\mu(s)\\
            =&\underset{s_0\sim \mu}{\mathbb{E}}[(1-\gamma)\sum_{t=0}^{\infty}\gamma^tPr^{\pi}(s_t=s|s_0)]\\
        \end{aligned}
    \end{equation}
    所以左边=右边。
\end{proof}
定义一个状态-动作多面体:\\
$\mathcal{K}_\mu :=\{d|d\geq 0\quad and \quad \sum_a d(s,a)=(1-\gamma)\mu (s)+\gamma\sum_{s',a'}P(s|s',a')d(s',a')\}$\\
可以看到,$\mathcal{K} _\mu$是所有可以的状态-动作分别的集合,即$d\in\mathcal{K} _\mu$当且仅当存在一个平稳的策略$\pi$使得$d^{\pi}_{\mu}=d$。\\
有了这些定义,对偶问题可以写成:\\
$\max\qquad\quad \frac{1}{1-\gamma}\sum_{s,a}d_\mu(s,a)r(s,a) $\\
subject to  $d\in \mathcal{K}_{\mu}$\\
目标函数可以看作是每个状态-动作对的密度乘上奖励,加权求和来求总奖励,求这个总奖励的最大值。解这个对偶LP求得一个解$d^*$,那么就可以求得最优策略:\\
$\pi^*(s,a)=\frac{d^*(s,a)}{\sum_{a'}d^*(s,a')}$\\
另一种求最优策略的方法是求$\underset{a}{argmax}\quad d^*(s,a')$。
    \section{Motivation}

In this section, we first illustrate the motivation of studying SDE from a machine learning perspective.
    \subsection{Approximating SGD}
First, we look at the SGD process:
\begin{equation}
    x^{n+1} = x^{n} - \eta_{k} \nabla \mathcal{L} (x^n; \xi^n),\tag{ SGD}
    \label{SGD}
\end{equation}
where the white noise $\xi^n$ characterize the randomness of the surrogate gradient in SGD method. Denote $\Sigma (x):=\mathbb{E}_{\xi}\left[\left( \nabla\mathcal{L}(x;\xi)- \nabla\mathcal{L}(x)\right)\left(\nabla\mathcal{L}(x;\xi)- \nabla\mathcal{L}(x)\right)^T  \right]   $ and \ref{SGD} writes:
$$
    x^{n+1} = x^{n} - \eta_{k} \nabla \mathcal{L} (x^n) + \sqrt{\eta \Sigma}  \sqrt{\eta} \mathcal{Z}^n, \mathcal{Z}^n\sim N(0,I_d).
$$
If we take the limit $\eta\rightarrow 0$ and regard $\sqrt{\eta} \mathcal{Z} ^n = \mathrm{d} W_t$, the SDE form of SGD is:
\begin{equation}
    \mathrm{d}X(t) = -\nabla\mathcal{L}(X(t))\mathrm{d}t+\sqrt{\eta \Sigma}\mathrm{d} W_t.\tag{ SDE-1}
    \label{SDE-1}
\end{equation}
Q: 
\begin{itemize}
    \item Is \ref{SDE-1} a good approximation of \ref{SGD}? 
    \item Good in what sense? 
    \item Is there a better one?
\end{itemize}
A: 
\begin{itemize}
    \item \ref{SDE-1} is a first-order weak approximation of \ref{SGD}. 
    \item Good in sense of testing: $\forall \left\lvert g(x)\right\rvert < K(1+\left\lvert x\right\rvert )^K , \left\lvert \mathbb{E} g(X(n\eta))-g(X^n)\right\rvert<C\eta^\alpha $
    \item There are higher order approximations!
\end{itemize}
For example, the second-order approximation of \ref{SGD} writes:
\begin{equation}
    \mathrm{d}X(t) = -\nabla \left(\mathcal{L}(X(t)) +\frac{\eta}{4}\left\lVert\nabla\mathcal{L}(X(t)) \right\rVert^2  \right)  \mathrm{d}t+\sqrt{\eta \Sigma}\mathrm{d} W_t.\tag{ SDE-2}
    \label{SDE-2}
\end{equation}
And another formulation (1-d Xiang) writes:
\begin{equation}
    \mathrm{d}X(t) = \frac{\log(1-\eta\mathcal{L}''(x))}{\eta\mathcal{L}''(x)}\mathcal{L}'(x) \mathrm{d}t + \sqrt{\frac{2\Sigma \cdot\log(1-\mathcal{L}''(x)\eta)}{\mathcal{L}''(x)(\mathcal{L}''(x)\eta-2)}}\mathrm{d}W_t.\tag{ SDE-Xiang-1-dim}
    \label{SDE-Xiang}
\end{equation}
The d-dimensional Xiang-Formulation is still under developing.
Another class of questions is follows:

Q: 
\begin{itemize}
    \item How are these more advanced flows derived?
    \item Why would the SDE approximation be useful?
\end{itemize}
, which will be answered in the following.





    \subsection{Langevin Dynamics}
Our goal of Langevin Dynamics is sampling from a Gibbs measure $\frac{e^{-\frac{\mathcal{L}(x)}{\sigma}}}{\mathcal{Z}_{\mathcal{L},\sigma}}$, where $\mathcal{Z}_{\mathcal{L},\sigma}$ is the normalizing constant.
The Langevin dynamics writes:
\begin{equation}
    \mathrm{d}X(t) = -\nabla\mathcal{L}(X(t))\mathrm{d}t+\sqrt{2\sigma}\mathrm{d} W_t.\tag{LD}
    \label{LD}
\end{equation}
Q: 
\begin{itemize}
    \item Why is this approach correct? I.e. why does LD have the correct equilibrium?
    \item How fast is the convergence?
\end{itemize}

The discrete-time version of \ref{LD} writes:
\begin{equation}
    X^{k+1} = X^k - \eta \nabla \mathcal{L}(X^k) + \sqrt{2\sigma}\sqrt{\eta}\mathcal{Z} ^k,\mathcal{Z}^n\sim N(0,I_d).\tag{LD-discrete}
\end{equation}

Q: 
\begin{itemize}
    \item What is the convergence property?
    \item Can we accelerate the convergence?
\end{itemize}


    \newpage
    \section{Ordinary Differential Equations}
To better understand the behavior of SDE, we can first take a look at its non-random counterpart, i.e., ODEs (Ordinary Differential Equations).


\begin{equation}
    \mathrm{d}X(t) = f(t,X(t)) \mathrm{d}t,X(0)=X_0\tag{ODE}
    \label{ODE}
\end{equation}

Example: Linear ODE, i.e., $ f(t,X(t)) = L X(t)$

\begin{equation}
    \begin{aligned}
        \frac{\mathrm{d}}{\mathrm{d}t}e^{-tL}X(t)
        &=-e^{-tL}LX(t)+e^{-tL}\frac{\mathrm{d}}{\mathrm{d}t}X(t)\\
       &=e^{-tL}\left( \frac{\mathrm{d}}{\mathrm{d}t}X(t)-LX(t) \right) =0\\
       &\Rightarrow e^{-tL}X(t)=e^{-tL}X(t)\vert_{t=0} =X_0\\
       &\Rightarrow X(t) = e^{tL}X_0\\
    \end{aligned}
\end{equation}

Next we look at the conception of Principle Flow proposed in \citet{rosca2023on}, consider minimizing the quadratic objective $f(x) = \frac{1}{2}x^THx$, assuming that $H$ is positive definite.

Consider the Gradient Descent Dynamic here:
$$
X^{n+1} = X^n-\eta\nabla\mathcal{L}(x) = X^n - \eta H X^n = (1-\eta H)X^n
$$
Therefore $X^{n} = (1-\eta H)^nX_0$, if we want to have $X^n = X(n\eta)$ for all $n$, we should have:
$$
(1-\eta H)^n = e^{n\eta L}\Rightarrow L = \frac{\log (1-\eta H)}{\eta}
$$
We obtain $\mathrm{d}X(t) = \frac{\log(1-\eta H)}{\eta}X(t)\mathrm{d}t$, which is the Principle Flow in the quadratic case. This can be generalized to the non-linear case:
$$
\mathrm{d}X(t) = \sum_{i = 1}^{d}  \frac{\log(1-\eta\lambda_i)}{\eta \lambda_i}\nabla\mathcal{L}(X(t))^T u_i\cdot u_i,
$$
where $\nabla^2\mathcal{L}(X(t)) = \sum_{i = 1}^{d}  \lambda_i u_iu_i^T$ is the SVD of $\nabla^2\mathcal{L}(X(t))$.
This generalization is derived in the sense of "backwards error analysis":
\begin{align}\label{backward error analysis}
    \left\{
     \begin{aligned}
         &\dot{\theta} = - \nabla\mathcal{L}(\theta) + \eta f_1(\theta) + \dots  + \eta ^nf_n(\theta) , \\
         &\theta^{n+1} = \theta^n -\eta \nabla\mathcal{L}(\theta^n), 
     \end{aligned}\right.
 \end{align}
We want to have $\theta^{n+1} = \theta(n\eta+\eta)$ and $\theta^n = \theta(n\eta)$


    \subsection{Numerical Solvers for ODE}
Write the ODE in the integral form:
$$
X(t+\Delta  t) = X(t) + \int_{t}^{t+\Delta t} f(X(\tau  ),\tau  )  \,\mathrm{d}\tau  
$$

Explicit Euler method:
$$
\overline{X} (t+\Delta  t)= X(t) + \int_{t}^{t+\Delta t} f(X(t  ),t  )  \,\mathrm{d}\tau  
$$

Implicit Euler method:
$$
X(t+\Delta  t)= X(t) + \int_{t}^{t+\Delta t} f(X(t+\Delta t),t+\Delta t)  \,\mathrm{d}\tau  
$$

Heun method:
$$
X(t+\Delta  t)=  X(t) + \frac{1}{2}\int_{t}^{t+\Delta t} f(X(t),t) + f(\overline{X}(t+\Delta t),t+\Delta t) \,\mathrm{d}\tau  
$$

Fourth order Runge-Kutta method:
\begin{align*}
&\Delta X_k^1 = f(\widehat{X}(t_k),t_k )\Delta t\\
&\Delta X_k^2 = f(\widehat{X}(t_k)+\Delta X_k^1/2,t_k+\Delta t/2 )\Delta t\\
&\Delta X_k^3 = f(\widehat{X}(t_k)+\Delta X_k^2/2,t_k+\Delta t/2 )\Delta t\\
&\Delta X_k^4 = f(\widehat{X}(t_k)+\Delta X_k^3,t_k+\Delta t )\Delta t\\
&\widehat{X}(t_{k+1}) = \widehat{X}(t_{k})+\frac{1}{6}(\Delta X_k^1+2\Delta X_k^2+2\Delta X_k^3+\Delta X_k^4)
\end{align*}
Order of approximation: $\left\lvert \widehat{X}(t_M)-X(t_M)\leq k\Delta t^P\right\rvert ,M = \frac{1}{\Delta t}$
    \subsection{Existence and Uniqueness of the solution to the ODE}
Picard iteration: start from the initial guess $\varphi_0(t)=X_0$,
recursively compute
$$\varphi_{n+1}(t) = X_0+\int_{t_0}^{t} f(\varphi_n(\tau),\tau) \,\mathrm{d}\tau. $$
If $f$ is continuous in both $x$ and $t$ and Lipschitz continuous in $x$, then:
$$
\lim_{n \to \infty}  \varphi_n(t) = X(t)
$$






    \newpage
    \section{Heuristic Derivation of SDE}
\subsection{Linear SDE}
For SDE, we assume $\mathrm{d}W_t \sim N(0,\mathrm{d}t)$ and SDE writes:
$$
\mathrm{d}X_t = FX_t\mathrm{d}t + \sqrt{\widehat{\Sigma } } \mathrm{d}W_t
$$
Then,
\begin{equation*}
    \begin{aligned}
        &\mathrm{d}\exp(-Ft)X_t\\
        =&-F\cdot\exp(-Ft)X_t\mathrm{d}t+\exp(-Ft)\mathrm{d}X_t\\
        =&\exp(-Ft)\sqrt{\widehat{\Sigma}}\mathrm{d}W_t\\
        \Rightarrow& \exp(-Ft)X_t = X_0+\int_{0}^{t}  \exp(-F\tau)\sqrt{\widehat{\Sigma}}\,\mathrm{d}W_\tau\\
        \Rightarrow& X_t = \exp(Ft)X_0+\exp(F(t-\tau))\sqrt{\widehat{\Sigma}}\,\mathrm{d}W_\tau
    \end{aligned}
\end{equation*}
So we know that $X_t$ remains Gaussian given $X_0 \sim N(m_0,P_0)$.
\begin{equation*}
    \begin{aligned}
        m_t=&\mathbb{E} X_t = \exp(F_t)m_0\\
        P_t=&\mathbb{E}\left[ (X_t-m_t)(X_t-m_t)^T \right] \\
        =&\exp(Ft)P_0\exp(Ft)^T + \int_{0}^{t} \exp(F(t-\tau))\widehat{\Sigma}\exp(F(t-\tau))^T \,d\tau
    \end{aligned}
\end{equation*}
, which reveals the property of OU process.






    \subsection{Informal derivation of Xiang's approach}
Consider the stochastic dynamics $(d=1)$:
$$
X^{n+1}=X^n-\eta\left( HX^n+\sqrt{\Sigma}Z^{n} \right) ,Z^{n}\sim N(0,I)
$$
so, $X^n$ remains Gaussian.
\begin{equation*}
    \begin{aligned}
        &X^{n+1}=\left(1-\eta H\right) X^n-\eta \sqrt{\Sigma}Z^{n} \\
        \Rightarrow& \frac{X^{n+1}}{\left(1-\eta H\right)^{n+1}} = \frac{X^{n}}{\left(1-\eta H\right)^{n}}-\frac{\eta \sqrt{\Sigma}Z^{n}}{\left(1-\eta H\right)^{n+1}}\\
        \Rightarrow& \frac{X^{n}}{\left(1-\eta H\right)^{n}} =X^0 - \eta\sqrt{\Sigma}\sum_{i = 1}^{n}\frac{Z^i}{\left(1-\eta H\right)^{i}} \\
        \Rightarrow& X^{n} =\left(1-\eta H\right)^{n}X^0 -\eta\sqrt{\Sigma}\sum_{i = 1}^{n}\left(1-\eta H\right)^{n-i}Z^i
    \end{aligned}
\end{equation*}
We can further calculate its mean and variance:
$$
m^n = \mathbb{E} X^n = (1-\eta H)^n \mathbb{E} X^0
$$
\begin{equation*}
    \begin{aligned}
        \mathbb{E}\left[ (x^n-m^n)(x^n-m^n)^T \right] =& \eta^2\Sigma  \sum_{i = 1}^{n}(1-\eta H)^{2(n-i)}\\
        =&\eta^2\Sigma \sum_{i = 0}^{n-1}(1-\eta H)^{2i}\\
        =&\eta^2\Sigma\frac{1-(1-\eta H)^{2n}}{1-(1-\eta H)^2}
    \end{aligned}
\end{equation*}






    \newpage
    \newcommand{\ito}{Ito }
\section{\ito Calculus and SDE}
\subsection{Stochastic Integral}

SDE should be understood as a shorthand of the stochastic integrated equation:
$$
X(t)=X(0)+\int_{0}^{t} f(\tau,X(\tau))\mathrm{d}\tau + \int_{0}^{t}L(\tau,X(\tau))\mathrm{d}W_\tau
$$
where the integral w.r.t. the Brownian motion should be understood as the limit:
$$
\int_{t_0}^{t}L(\tau,X(\tau))\mathrm{d}W(\tau) = \lim_{n \to \infty}  \sum_{k}L(t_k,X(t_k))\left[ W(t{k+1})-W(t_k) \right] 
$$
where $t_0<t_1<\dots<t_n = t$ is a partition of $[0,t]$ and $\min t_{i+1} - t_i \to 0$ as $n\to \infty$.
We would not make this definition rigorous in this overview. This would be the main objective of this course.



    \subsection{\ito's formula}
We directly assume the \ito's formula to be true.
For SDE: $\mathrm{d}X_t = -(t,X_t)\mathrm{d}t + L(t,X(t))\mathrm{d}W_t$



\begin{theorem}
(\ito formula). Assume that $X(t)$ is an \ito process, and consider an arbitrary (scalar) function $\phi(X(t),t)$ of the process. Then the \ito differential of $\phi $, that is, the \ito SDE for $\phi$ is given as
\begin{equation*}
    \begin{aligned}
        \mathrm{d}\phi &= \frac{\partial \phi}{\partial t}+\sum_{i}\frac{\partial\phi}{\partial x_i}\mathrm{d}x_i+\frac{1}{2}\sum_{i,j}(\frac{\partial^2\phi}{\partial x_i\partial x_j})\mathrm{d}x_i \mathrm{d}x_j\\
        &=\frac{\partial\phi}{\partial t}\mathrm{d}t+(\nabla\phi)^T\mathrm{d}x+\frac{1}{2}\nabla^2\phi:LL^T\mathrm{d}t
    \end{aligned}
\end{equation*}
where $A:B = tr(A^TB)$
\end{theorem}
As a sanity check, consider $X(t) = W(t)$ (i.e. $f\equiv 0,L=I$) and $\phi(x) = \frac{1}{2}x^2$, then
$$
\mathrm{d} \phi(X(t)) = X(t)\cdot \mathrm{d}W(t)+\frac{1}{2}\mathrm{d}t = W(t)\mathrm{d}W(t)+\frac{1}{2}\mathrm{d}t
$$

    \subsection{Uniqueness and existence}
If both $f$ and $L$ grow at most linearly w.r.t. $x$ and are Lipschitz continuous w.r.t. $x$, the solution is unique.

    \newpage
    \section{Evolution of the Distribution}
\subsection{Forward and Backward Kolmogorov's equation}
Fokker-Planck equation: Let $p(x,t)$ be the density of $X(t)$, where $X(0) \sim p(x,0)$ and
$$
\mathrm{d}X(t) = f(t,X(t))\mathrm{d}t + \sqrt{2\Sigma(t,X(t))}\mathrm{d}W_t
$$ 
We have:
$$
\frac{\partial}{\partial t}p+\nabla\cdot(pf) = \nabla\cdot(\nabla\cdot \Sigma p)
$$
where for a matrix field, $\nabla\cdot$ is applied row-wisely.
\begin{proof}
Consider a twice differentiable function $\phi$.
$$
\mathbb{E}[\phi(X_t)]=\int \phi(x)p(t,x)\mathrm{d}x
$$ 
Using \ito's Lemma, we have:
$$
\mathrm{d}\phi(X_t) = (\nabla\phi\cdot f+\nabla^2\phi:\Sigma)\mathrm{d}t+\nabla\phi\cdot\sqrt{2\Sigma}\mathrm{d}W_t
$$
\begin{equation*}
    \begin{aligned}
        \mathbb{E}\left[\phi(X_t)\right]=& \mathbb{E} \left[ \phi(X_0) + \int_{0}^{t}(\nabla\phi\cdot f+\nabla^2\phi:\Sigma)\mathrm{d}\tau +\int_{0}^{t} \nabla\phi\cdot\sqrt{2\Sigma}\mathrm{d}W_\tau \right] \\
        =&\mathbb{E}\left[\phi(X_0)\right] + \int_{0}^{t}\mathbb{E}(\nabla\phi\cdot f+\nabla^2\phi:\Sigma)\mathrm{d}\tau
    \end{aligned}
\end{equation*}
\begin{equation*}
    \begin{aligned}
        \frac{\mathrm{d}}{\mathrm{d}t}\mathbb{E}\left[\phi(X_t)\right]=& \mathbb{E}_{x\sim p(t,x)}\left[  \nabla\phi(x)\cdot f(t,x)+\nabla^2\phi(x):\Sigma(t,x)\right] \\
        =&\int \nabla\phi(x)\cdot f(t,x)\cdotp(t,x)+\nabla^2\phi(x):\Sigma(t,x)\cdotp(t,x)  \,\mathrm{d}x\\
        =&\int -\nabla\cdot(p\cdot f)\cdot \phi + \phi\cdot\nabla\cdot(\nabla\cdot\Sigma p) \,\mathrm{d}x\\
        =&\int\phi(x)\cdot\frac{\partial}{\partial t}p(t,x)\mathrm{d}x
    \end{aligned}
\end{equation*}
Since $\phi$ is chosen arbitrarily, we have
$$
\frac{\partial}{\partial t}p+\nabla\cdot(pf) = \nabla\cdot(\nabla\cdot \Sigma p)
$$
\end{proof}





    \subsection{Deriving Xiang's approach rigorously}














    \subsection{Convergence analysis of Langevin dynamics.}

    % \newpage
    % %\section{Computational Complexity}
我们定义$L,(P,r,\gamma)$是确定一个MDP所需的bit-size,假定算术运算+,-,$\times$,$\div$都使用单位时间(unit time)。我们希望找到一个算法,它能找到在$L,(P,r,\gamma)$、$|mathcal{S}|$、$|mathcal{A}|$的多项式时间内找到最优策略。
如果一个算法能在$|S|$、$|A|$的多项式时间(与$L,(P,r,\gamma)$无关)内找到最优策略,则称该算法为strongly polynomial。
\subsection{Value Iteration}
\begin{lemma}[收缩性]
    对任意两个向量$Q$,$Q'\in \mathbb{R}^{|mathcal{S}||mathcal{A}|}$,$\Vert \tau Q-\tau Q' \Vert_\infty\leq\gamma\Vert Q-Q' \Vert_\infty$。
    这说明了$\tau$是一个收缩映射。
 \end{lemma}
\begin{proof}
    首先证明对于所有的$s\in \mathcal{S},|V_Q(s)-V_{Q'(s)}|\leq \underset{a\in \mathcal{A}}{max}|Q(s,a)-Q'(s,a)|$。\\
    不妨设$V_Q(s)>V_{Q'}(s)$(另一半是对称的),记$a^*=\underset{a}{argmax}Q(s,a)$。\\
    $|V_Q(s)-V_{Q'(s)}|=\underset{a\in\mathcal{A}}{max}Q(s,a)-\underset{a\in\mathcal{A}}{max}Q'(s,a)\leq Q(s,a^*)-Q'(s,a^*)\leq\underset{a\in\mathcal{A}}{max}|Q(s,a)-Q'(s,a)|$
    \begin{equation}
        \begin{aligned}
            \Vert \tau Q-\tau Q' \Vert_\infty &= \Vert r+\gamma PV_Q-r-\gamma PV_{Q'} \Vert_\infty\\
            &= \gamma \Vert P(V_Q-V_{Q'}) \Vert_\infty\\
            &\leq \gamma \Vert V_Q-V_{Q'} \Vert_\infty\\
            &= \gamma \underset{s}{\max}|V_Q-V_{Q'}|\\
            &\leq \underset{s}{\max}\underset{a}{\max}|Q(s,a)-Q'(s,a)|\\
            &= \gamma\Vert Q-Q' \Vert_\infty
        \end{aligned}
    \end{equation}
\end{proof}
由该引理可见,经历了k次迭代后,$\Vert Q_k-Q^* \Vert_\infty\leq\gamma^k\Vert Q_0-Q^*\Vert_\infty$,因此$\underset{k\to \infty}{\lim}Q_k-Q^*=0$,算法收敛。

\begin{lemma}[Q-Error Amplification)]
    对任意向量$Q\in \mathbb{R}^{|mathcal{S}||mathcal{A}|},V^{\pi_Q}\geq V^*-\frac{2\Vert Q-Q^*\Vert_\infty}{1-\gamma}\mathbbm{1}$
    \\其中$\mathbbm{1}$为全1向量。
 \end{lemma}
\begin{proof}
    对于固定的$s$以及$a=\pi_Q(s),(\pi_Q(s)=\underset{a\in \mathcal{A}}{argmax}Q(s,a))$
    \begin{equation}
        \begin{aligned}
            V^*(s)-V^{\pi_Q}(s)&=Q^*(s,\pi^*(s))-Q^{\pi_Q}(s,a)\\
            &=Q^*(s,\pi^*(s))-Q^*(s,a)+Q^*(s,a)-Q^(\pi_Q)(s,a)\\
            &=Q^*(s,\pi^*(s))-Q^*(s,a)+\gamma \mathbb{E}_{s'\sim P(\cdot |s,a)}[V^*(s')-V^{\pi_Q}(s')]\\
            &\leq Q^*(s,\pi^*(s))-Q(s,\pi^*(s))+Q(s,a)-Q^*(s,a)+\gamma \mathbb{E}_{s'\sim P(\cdot |s,a)}[V^*(s')-V^{\pi_Q}(s')]\\
            &\leq 2\Vert Q-Q^* \Vert_\infty+\gamma\Vert V^*-V^{\pi_Q}\Vert_\infty
        \end{aligned}
    \end{equation}
\end{proof}

\begin{theorem}[Q-value iteration convergence]
    设$Q^{(0)}=0,Q^{(k+1)}=\tau Q^{(k)},k=0,1,\ldots \pi^{(k)=pi_{Q^{(k)}}}$,当$k\geq \frac{\log \frac{2}{(1-\gamma )^2\epsilon}}{1-\gamma}$,\\
    $V^{\pi^{(k)}}\geq V^*-\epsilon \mathbbm{1}$,即k次迭代后$V^{\pi^{(k)}}$和$V^*$非常接近。
\end{theorem}
\begin{proof}
    $\Vert Q^{(k)}-Q^*\Vert_\infty=\Vert \tau ^kQ^{(0)}-\tau ^kQ^*\Vert_\infty \leq\gamma^k\Vert Q^{(0)}-Q^*\Vert_\infty=(1-()1-\gamma)^k\leq \frac{exp(-(1-\gamma)k)}{1-\gamma} $
\end{proof}

\subsection{Value Iteration}
策略迭代算法从任意一个策略$\pi_0$出发,并对k=0,1,2,$\ldots $重复接下来的两步:
1.策略评估。计算$Q^{\pi_k}$\\
2.策略提升。更新策略:$\pi_{k+1}=\pi_{Q^{\pi_k}}$,即$\pi_{k+1}(s)=\underset{a\in\mathcal{A}}{argmax}Q^{\pi_k}(s,a)$

\begin{lemma}
    1.$Q^{\pi_{k+1}}\geq \tau Q^{\pi_k}\geq Q^{\pi_k}$\\
    2.$\Vert Q^{\pi_{k+1}}-Q^* \Vert_\infty\leq\gamma\Vert Q^{\pi_k}-Q^* \Vert_\infty$
\end{lemma}
\begin{proof}
    首先证明$\tau Q^{\pi_k}\geq Q^{\pi_k}$。注意到策略迭代中的策略都是确定策略,䫅对于所有$k,s$,$V^{\pi_k}(s)=Q^{\pi_k}(s,\pi_k(s))$。\\
    \begin{equation}
        \begin{aligned}
            \tau Q^{\pi_k}(s,a)&=r(s,a)+\gamma \mathbb{E}_{s'\sim P(\cdot |s,a)}[\underset{a'}{\max}Q^{\pi_k}(s',a')]\\
            &\geq r(s,a)+\gamma\mathbb{E}_{s'\sim P(\cdot |s,a)}[Q^{\pi_k}(s',\pi_k(s'))]=Q^{\pi_k}(s,a)
        \end{aligned}
    \end{equation}
    \\再证明$Q^{\pi_{k+1}}\geq \tau Q^{\pi_k}$,这需要先证明$Q^{\pi_{k+1}}\geq Q^{\pi_k}$:\\
    $Q^{\pi_k}=r+\gamma P^{\pi_k}Q^{\pi_k}\leq r+\gamma P^{\pi_{k+1}}Q^{\pi_k}\leq \sum_{t=0}^{\infty}\gamma^t(P^{\pi_{k+1}})^tr=Q^{\pi_{k+1}}$\\
    第一个不等式是因为$\pi_{k+1}$是greedy policy,所以一定比$\pi_{k}$更好,第二个不等式由递归得到。\\因此
    \begin{equation}
        \begin{aligned}
            Q^{\pi_{k+1}}(s,a)&=r(s,a)+\gamma \mathbb{E}_{s'\sim P(\cdot |s,a)}[Q^{\pi_{k+1}}(s',\pi_{k+1}(s'))]\\
            &\geq r(s,a)+\gamma \mathbb{E}_{s'\sim P(\cdot |s,a)}[Q^{\pi_{k}}(s',\pi_{k+1}(s'))]\\
            &=r(s,a)+\gamma \mathbb{E}_{s'\sim P(\cdot |s,a)}[\underset{a'}{\max}Q^{\pi_k}(s',a')]=\tau Q^{\pi_k}(s,a)
        \end{aligned}
    \end{equation}
    (1)式证明完成。
    下面证明(2)式:\\
    $\Vert Q^*-Q^{\pi_{k+1}} \Vert_\infty\leq\Vert Q^*-\tau Q^{\pi_{k}} \Vert_\infty =\Vert \tau Q^*-\tau Q^{\pi_{k}} \Vert_\infty\leq\gamma\Vert Q^*-Q^{\pi_k} \Vert_\infty$\\
    证明完成。
\end{proof}

\begin{theorem}[Policy iteration convergence]
    设$Q^{\pi_0}=0,\pi_0$为任意初始策略。当$k\geq \frac{\log\frac{1}{(1-\gamma )\epsilon}}{1-\gamma} $,第k个策略有这样的performance bound:\\
    $Q^{\pi_k}\geq Q^*-\epsilon $
\end{theorem}

\begin{proof}
    \begin{equation}
        \begin{aligned}
            \Vert Q^*-Q^{\pi_k}\Vert_\infty&\leq\gamma\Vert Q^*-Q^{\pi_{k-1}}\Vert_\infty
            &\leq \gamma^k\Vert Q^*-Q^{\pi_0}\Vert_\infty
            &=\gamma^k\Vert Q^\Vert_\infty
            &=(1-(1-\gamma))^k\Vert Q^\Vert_\infty
            &\leq \frac{\exp (-(1-\gamma)k)}{1-\gamma} 
        \end{aligned}
    \end{equation}
\end{proof}


\subsection{Linear Programming Approach}
线性规划的方法可以在严格多项式时间内解决问题。\\
\subsubsection{原始问题}
最初的想法是求解\\
$\qquad\underset{V\in \mathbb{R}^{|\mathcal{S}|}}{\min}\qquad{\sum_s\mu (s)V(s)}$\\
subject to  $V(s)\geq\underset{a\in \mathcal{A}}{\max}[r(s,a)+\gamma\underset{s'}{\sum}{P(s'|s,a)V(s')}] \qquad \forall s\in\mathcal{S}$\\
但这不是LP问题。将其转化为LP问题,得到:\\
$\qquad\underset{V\in \mathbb{R}^{|\mathcal{S}|}}{\min}\qquad{\sum_s\mu (s)V(s)}$\\
subject to  $V(s)\geq r(s,a)+\gamma\underset{s'}{\sum}{P(s'|s,a)V(s')} \qquad \forall a\in\mathcal{A},s\in\mathcal{S} $\\
其中,$\mu (s)$是初始状态分布。如果$\mu$ has full support,那么该问题的唯一解就是$V^*(s)$。
\begin{proof}
    证明利用了$\tau$的单调性,即$V \geq V^{'}$时,有$\tau V \geq \tau V^{'}$。\\
    令$V'=\tau V$,则$\tau V\geq\tau V'=\tau^2V$。迭代得到:$V\geq\tau^\infty=V^*$。\\
    因此该约束条件下得到的解都是$V\geq V^*$的情况,由于目标函数是求$\qquad\underset{V\in \mathbb{R}^{|\mathcal{S}|}}{\min}\qquad{\sum_s\mu (s)V(s)}$,最终得到的解即$V=V^*$。
\end{proof}
\subsubsection{对偶问题}
对每一个LP都存在一个对偶问题,原问题的决策变量对应对偶问题的约束条件,原问题的约束条件对应对偶问题的决策变量。\\
对于固定的策略$\pi$,定义关于状态和动作的visitation measure:
\begin{equation}
    d^{\pi}_{s_0}(s,a)=(1-\gamma)\sum_{t=0}^{\infty}\gamma^tPr^\pi (s_t=s,a_t=a|s_0)
\end{equation}
其中$Pr^\pi (s_t=s,a_t=a|s_0)$是从状态$s_0$出发,经过策略$\pi$,到达$s_t=s,a_t=a$的概率。并记$d^{\pi}_{\mu}(s,a)=\mathbb{E}_{s_0\sim\mu}[d^{\pi}_{s_0}(s,a)]$\\
对于任意的状态$s$有:
\begin{equation}
    \underset{a}{\sum}d^{\pi}_{\mu}(s,a)=(1-\gamma)\mu(s)+\gamma\underset{s',a'}{\sum}P(s|s',a')d^{\pi}_{\mu}(s',a')
\end{equation}

\begin{proof}
    左边:\\
    \begin{equation}
        \begin{aligned}
            \underset{a}{\sum}d^{\pi}_{\mu}(s,a)&=d^{\pi}_{\mu}(s)\\
           &=\underset{s_0\sim \mu}{\mathbb{E}}[d^{\pi}_{s_0}(s)]\\
           &=\underset{s_0\sim \mu}{\mathbb{E}}[(1-\gamma)\sum_{t=0}^{\infty}\gamma^tPr^{\pi}(s_t=s|s_0)]
        \end{aligned}
    \end{equation}
    右边:
    \begin{equation}
        \begin{aligned}
            &\gamma\underset{s',a'}{\sum}P(s|s',a')d^{\pi}_{\mu}(s',a')+(1-\gamma)\mu(s)\\
            =&\gamma\underset{s',a'}{\sum}P(s|s',a')\underset{s_0\sim \mu}{\mathbb{E}}[(1-\gamma)\sum_{t=0}^{\infty}\gamma^tPr^{\pi}(s_t=s',a_t=a'|s_0)]+(1-\gamma)\mu(s)\\
            =&\underset{s_0\sim \mu}{\mathbb{E}}[(1-\gamma)\sum_{t=0}^{\infty}\sum_{s',a'}\gamma^{t+1}P(s|s',a')Pr^{\pi}(s_t=s',a_t=a'|s_0)]+(1-\gamma)\mu(s)\\
            =&\underset{s_0\sim \mu}{\mathbb{E}}[(1-\gamma)\sum_{t=0}^{\infty}\sum_{s',a'}\gamma^{t+1}Pr^{\pi}(s|s_0)]+(1-\gamma)\mu(s)\\
            =&\underset{s_0\sim \mu}{\mathbb{E}}[(1-\gamma)\sum_{t=0}^{\infty}\gamma^tPr^{\pi}(s_t=s|s_0)]\\
        \end{aligned}
    \end{equation}
    所以左边=右边。
\end{proof}
定义一个状态-动作多面体:\\
$\mathcal{K}_\mu :=\{d|d\geq 0\quad and \quad \sum_a d(s,a)=(1-\gamma)\mu (s)+\gamma\sum_{s',a'}P(s|s',a')d(s',a')\}$\\
可以看到,$\mathcal{K} _\mu$是所有可以的状态-动作分别的集合,即$d\in\mathcal{K} _\mu$当且仅当存在一个平稳的策略$\pi$使得$d^{\pi}_{\mu}=d$。\\
有了这些定义,对偶问题可以写成:\\
$\max\qquad\quad \frac{1}{1-\gamma}\sum_{s,a}d_\mu(s,a)r(s,a) $\\
subject to  $d\in \mathcal{K}_{\mu}$\\
目标函数可以看作是每个状态-动作对的密度乘上奖励,加权求和来求总奖励,求这个总奖励的最大值。解这个对偶LP求得一个解$d^*$,那么就可以求得最优策略:\\
$\pi^*(s,a)=\frac{d^*(s,a)}{\sum_{a'}d^*(s,a')}$\\
另一种求最优策略的方法是求$\underset{a}{argmax}\quad d^*(s,a')$。
    % \section{简介}

本文的目的在于使读者对ParVI采样方法形成初步了解,
全文将从欧几里得空间上的梯度流入手,随后介绍概率空间上的梯度流。
之后介绍ParVI方法,以及本研究组提出的DPVI框架。
最后还会补充mirror DPVI方法的介绍。
\par
预备知识:凸分析,测度论,泛函分析,多元微积分
\par
本人学习整理若有疏忽,发现错误还望指正,
请发送邮箱\href{wangfangyikang@zju.edu.cn}{wangfangyikang@zju.edu.cn}
    % \newpage
    % 

\section{在欧几里得空间上的梯度流}
为了引出概率空间上的梯度流,本节首先从梯度下降视角引出欧几里得空间上的梯度流。
随后将欧几里得空间上的思路推广至概率空间。
考察一个经典的优化问题:
$$ \underset{x \in \mathbb{R}^{n}}{min} f(x) ,f:\mathbb{R} ^n \mapsto \mathbb{R} $$
\par
在此问题上,有经典的梯度下降算法(Gradient Descent Algorithm):
\begin{equation}
    x_{k+1} = x_{k} - \eta_{k} \nabla f(x_{k}) %\label{GD0}
\end{equation}
其中$\eta_{k}$为第$k$步迭代的步长,
$\nabla f(x_{k})$为$f(x)$在$x_{k}$这一点上的梯度(假设$f(x)$的性质足够好)。
以上的迭代运算可被重写为:
$$ \frac{x_{k+1}-x_{k}}{\eta_{k}} =  -\nabla f(x_{k})$$
上式可以看作如下ODE方程的显式欧拉插值算法每一步的迭代操作。

\begin{equation}
    \frac{dX(t)}{dt} =  -\nabla f(x_{k}) %\label{GD0}
\end{equation}

由此,梯度下降算法可以看作是欧几里得空间上的梯度流的离散插值形式。
现在先简单证明一下此梯度流可以保证,
在$f:\mathbb{R}^{n}\rightarrow\mathbb{R}$为凸函数时,随机过程$X(t)$是趋近于最优点的。
\begin{proof}
    令 $     g(X(t)=\frac{1}{2}\Arrowvert X(t)-x^* \Arrowvert^2)$,其中$x^*=argminf(x)$,则有:
    \begin{equation}
        \begin{aligned}
            \frac{dg(X(t))}{dt}&=\frac{dg(X(t))}{dX(t)}\frac{dX(t)}{dt}\\
           &=\langle  \frac{dX(t)}{dt},X(t)-x^*  \rangle\\
           &= - \langle  \nabla_x f(X(t)),X(t)-x^*  \rangle\\
           &=  \langle  \nabla_x f(X(t)),x^*-X(t)  \rangle\\
           &\leqslant f(x^*)-f(X(t))\\
           &\leqslant 0
        \end{aligned}
    \end{equation}
    第一个不等式来自于$f(x)$的凸性质。\newline
    可见$g(X(t))$随着时间$t$减小,$X(t)$逐步逼近$x^*$。
\end{proof}

\par
让我们再一次审视$\frac{dX(t)}{dt} =  -\nabla f(x_{k}) $,这是一个处处由$-\nabla f(x)$ 定义的向量场。
回顾多元微积分中梯度的概念,梯度存在的时候可以由
$\nabla f(x) = [ \frac{\partial  f(\cdot )}{\partial  x_1} 
,\dots,\frac{\partial  f(\cdot )}{\partial  x_n}] $
进行计算,但这并不是梯度这一概念的定义,更不是梯度这一概念的本质。
接下来我将先介绍梯度的定义,而后介绍梯度的本质。
只有理解了欧几里得空间中梯度的深层内涵,才有可能理解概率空间上的变分(variation)操作。

\par
以$f:\mathbb{R} ^n\mapsto \mathbb{R} $为例,首先定义方向导数
\begin{definition}
    函数$f$在$\widehat{x} $这一点处关于方向$h\in \mathbb{R}^n$的方向导数定义为:
    \begin{equation}
        f'(\widehat{x},h) = \lim_{t\to 0}\frac{f(\widehat{x}+th)-f(\widehat{x})}{t}  
    \end{equation}
\end{definition}

接下来给出加托可微性和梯度的定义。
\begin{definition}
    当方向导数$f'(\widehat{x},h) ,h\in \mathcal{X} $关于$h$是一个线性系统(满足数乘保持性和加法保持性)时,即
    $f'(\widehat{x},h) = \mathcal{A} (h)$(其中$\mathcal{A} $是某个线性算子)时,
    我们说$f$在$\widehat{x}$这一点加托-可微。
\end{definition}

根据泛函分析中的里茨表示定理(Riesz representation theorem),$\mathbb{R} ^n$
空间上的线性算子构成的对偶空间与原始空间同构。考察$(\mathbb{R} ^n,<\cdot,\cdot>_2)$这一Hilbert空间,
则对于任意一个光滑的线性映射$l:\mathbb{R} ^n \mapsto \mathbb{F} \in \{\mathbb{R} ,\mathbb{C} \} $,
有唯一一个$x_{l } \in \mathbb{R} ^n $,使得$\forall x \in \mathbb{R} ^n ,l(x) = \langle x_{l } ,x\rangle $。
\par
所以$f:\mathbb{R} ^n\mapsto \mathbb{R} $在某一点的梯度实际上是将加托-可微线性算子写作了一个向量,
梯度的内涵是$f$在$\widehat{x}$邻域内的线性近似。回忆一下,这和我们当年学习的一阶泰勒展开是吻合的。
$$f(\widehat{x}+h) \approx f(x)+\mathcal{A} (h)+\mathcal{O} (\parallel h\parallel )$$

其中$\mathcal{A}$是一个光滑的线性算子,重写线性算子为内积形式,记作:
$$f(\widehat{x}+h) \approx f(x)+\left\langle  \nabla f(\widehat{x}),h \right\rangle +\mathcal{O} (\parallel h\parallel )$$

\par 
从这一角度对梯度进行理解之后,我们可以模仿梯度的定义来定义概率分布的变分,进一步地,将梯度流拓展到概率空间,。

    \clearpage
    \bibliography{main}


\end{document}